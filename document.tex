\documentclass[10pt,twocolumn]{article}

% use the oxycomps style file
\usepackage{oxycomps}

% usage: \fixme[comments describing issue]{text to be fixed}
% define \fixme as not doing anything special
\newcommand{\fixme}[2][]{#2}
% overwrite it so it shows up as red
\renewcommand{\fixme}[2][]{\textcolor{red}{#2}}
% overwrite it again so related text shows as footnotes
%\renewcommand{\fixme}[2][]{\textcolor{red}{#2\footnote{#1}}}

% read references.bib for the bibtex data
\bibliography{reference}

% include metadata in the generated pdf file
\pdfinfo{
    /Title (Junior Seminar '25)
    /Author (Seolbin Hong)
}

% set the title and author information
\title{Junior Seminar '25}
\author{Seolbin Hong}
\affiliation{Occidental College}
\email{shong4@oxy.edu}

\begin{document}

\maketitle



\section{Introduction and Problem Context}
Games can influence and evoke people's emotions. It can make them feel competitive, sad, happy and bring a biological, physical sense to the user. While movies and shows allowed listeners to immerse themselves through eye and emotion, games have allowed viewers to interact with a new world physically, and consequently, also be interacted with. As new genres and technology have formed to take part in this industry, cozy games, in particular, have gained rapid popularity during and after the pandemic. The pandemic has led to an overall increase in attention to self-care, as well as relaxation, which people have chosen to seek in games\cite{Pandemic}. Many players have been seen to be engaged with and appreciative of less stressful environments, unlike other games that focus primarily on high-stakes and fast-paced interactions with studies showing that games can help with anxiety and feelings of loneliness.\cite{Cozy} Remiss of the popularity of cozy games and its nature of relaxation, the implementation of haptics have not been seen as heavily in them compared to action genres. 

Haptics is the utilization of technology with the sense of touch and feel. This can be seen through vibrations, visual cues, auditory cues, and more. As the game industry gains popularity, newer and more advanced technologies are developed to help enhance the user experience and immersion. This can be seen with the recent versions of the PS5 controllers that allow adaptive triggers, where developers can control the resistance of the triggers for certain actions. It also has a touch pad that you can use to perform actions and interact with different objects. Though this is unique to the PS5 and has created opportunities for game companies to use, it has not been fully implemented by the industry. 

Action games require quicker game play and  often employ advanced haptic feedback and controls to provide immersion with a need for all of the players' attention, requiring live-time responses and rewarding it by providing live-time feedback. With haptics, this feedback has become beyond auditory and visual, but also physical. Cozy games, especially, do not utilize these as in depth or as intensely. This could be due to an attempt to maintain the game's environment of serenity and feeling of choice. However, I would like to challenge this notion of simplicity with haptic feedback for cozy games and see how we can increase immersion without disrupting the game's genre and goal. This paper will examine how haptics have been portrayed in games, specifically in cozy games, and how they have contributed to the goals of the games. This paper will also explore the ethical concerns regarding cozy sandbox games.
\section{Background}

\subsection{Technical Background on Haptics}

All games use sounds and visuals to help provide immersion. However, when given the opportunity to give a hands-on experience like Virtual Reality (VR) and controllers, we can give the user a physical response from their choices. Abdulmotaleb El Saddik says that we can roughly categorize haptic interfaces into two categories: “force feedback devices" and "tactile devices”. The former is of force/torque that allows the experience of resistance and roughness, while the other is more of the feeling on the skin like vibration and temperature. He also emphasizes the effects of good visuals and spatial biases within games to provide immersion \cite{saddik_haptics_2012}. I would like to focus on and think about the extension of applying resistance in cozy games. 

Although a key point in cozy games may be its leisure activities, resistance can help them feel more rewarded for playing and completing certain activities. Saddik says that with the more interactions a game has, the more immersed the player may be due to the diverse environment. When we compare cozy games to action games, we may see that action games offer more variety in the types of interaction they can do, while "cozy" games are considered more in tune with patterns and repetition, but both draw in high participation. For example, with a game like \textit{Valorant}\cite{valorant2020}, even though it has a limited set of terrains and weapons, the variety of interactions comes from the randomized players with whom the user plays, with varying skill levels and action preferences. With a set of actions that all players have access to, this type of game  encourages users to create an environment, in which they engage with new behavior and strategies to achieve their goal of winning. They must be quick, creative, and immersed to create actions that would beat the other players. This is different compared to a game like \textit{Animal Crossing}\cite{animalcrossing2020}, where even with randomized Non-Player Characters and interactions with other users, there is not as great a competitive need to be creative and find new interactions.

We can also see that there are different devices that were used to help provide a physical sense of gaming. The early controllers in the 1960s used toggle switches to help mimic a sense of action from inside the game to the real world.\cite{cummings_evolution_2012} Then in the late 1970s and early 1980s, we saw the development of a joystick with working compatibilities with multiple systems. This led to more developments with buttons and directional pads as a compact alternative to joysticks. And then, we created more portable controllers, adding shoulder buttons to help diversify interface for games, and saw a rise in analog sticks and vibration feedback for more nuanced control and responsive tactile sensations. Since then, we have introduced motion sensing and VR technologies to help with our sensation of immersion, as well as haptic feedback. However, many use keyboard and controllers to engage with games in their homes.

There are also different game engines that can be used to explore haptics within games like Unity, Unreal, and Godot. However, this literary review will mainly refer to Unity, as it had more support for the processes and controller interface that best fit the requirements for the Methods and Evaluations for the technical project. There will also be an emphasis on visuals and sound for Senior Comps to understand the balance of immersion in the game environment based on the intended goal of the project.

\subsection{Coziness as a Genre}

The gaming industry is growing with users and products. The number of video gamers has increased from 7. 5\% in 2022 and we expect to see this number reach 13\% by 2029.\cite{kody_2024} When looking at another statistic from The Entertainment Software Association (ESA) on why users play games, 68\% said that they play for relaxation. And with the pandemic opening new doors for people to explore relaxation and self-care, the number continues to grow.

The number of players is not the only thing that is growing. This genre has allowed less traditional users to create and diversify communities that help relieve stress and produce creative outlets. We can see that more women and older generations take part in games, as we see the statistics improve regarding gender and the average age of a video game player increasing from 32 in 2023 to 36 in 2024\cite{kody_2024}. As the industry grows with popularity and money, it brings to question how technical developments will grow as well.

Like mentioned, haptic feedback in cozy games is often more tame. This could potentially be due to developers' wish to not disrupt the serenity in their environments.\cite{Cozy} When we look at games like \textit{Animal Crossing: New Horizons} or \textit{Stardew Valley}, we can see that these games emphasize relaxation, emotional connection, and slow progression that is often controlled primarily by the user. In these games, they may give a slight vibration when picking up an object or performing a specific act like fishing. This design decision reflects the intent of cozy games, which is to provide immersion and sensory experience to their players while also doing it in a way that does not take away from the game's core themes of tranquility. Activated feedback supports gameplay elements that support the user's control of their environment and experience, an important aspect that people may gravitate toward after experiencing the pandemic\cite{Pandemic}. Gentle haptic feedback can also be seen to mirror real-world tasks that people perform to calm their anxieties, and is aimed at reinforcing the comforting and repetitive nature of these \cite{pub:74942}.

However, with action games, haptic feedback is crucial in enhancing the player's physical connection to the game world, aiding in their goals of achievements by invoking quicker real-time responses, especially against other players. Specifically for action games, haptic feedback can increase immersion and make the experience multi-sensory to mimic the real world.\cite{SöDerströM_Larsson_Lundqvist_Norberg_Andersson_Mejtoft_2022} For example, the sensation of a weapon recoiling when fired in a game can help simulate the physical feedback that would happen in real life, reinforcing a type of high-energy, tension-filled environment of these games. A sudden rumble in the controller due to an explosion or impact of a melee strike can also increase the emotional intensity of a moment and encourage the player to be more immersed in the game to be aware of all that is happening. Like in real life, the more aware you are, the quicker you can respond. This can also apply to games. 

The differences between the haptic feedback in cozy games and action games may be dependent on the competitive nature and goals of the game. The desired immersive haptic feedback for action games may seem overly aggressive for cozy games, disrupting the player's sense of serenity. To compare, action games may attempt to make you feel in tune with your character's movements, while cozy games attempt to make you feel in tune with the game's environment. Action games evoke more surprises and consequentially more variety in haptic feedback, while cozy games are more routine-based, so there might be less abrupt surprises and aggressiveness in the control. However, haptic feedback can help elevate the experience in all genres.\cite{saddik_haptics_2012} Beyond immersion, by providing more cues for the user, it can help increase accessibility for players as well.\cite{Ganis_Gulli_Fontana_Serafin_2023}

\subsection{Prior Works}
The origins of the "cozy game" genre can be traced back to the late 1900s with \textit{Little Computer People} and \textit{Harvest Moon}. \textit{Little Computer People} was released in 1985 and introduced the user to a character that lived in a small house. This could be one of the earliest examples of life simulation games/ \textit{Harvest Moon} in particular has served as an inspiration for games like \textit{Stardew Valley} with its art style and game-play of farming simulation. It allowed for players to cultivate crops, raise animals, and also interact with its rural environment. It raised an emphasis on routine and community, which many cozy games can be seen to follow today.

The genre has gained more popularity with \textit{Animal Crossing}\cite{animalcrossing2020} in 2020 due to the pandemic. It utilized haptics to help players perform more accurately and successfully in collecting items, as well as provide a more immersive experience. The game is set in an island that you move into as a new resident. You can collect items through shaking trees, picking up objects, chopping wood, mining and through the shops and residents on the island. You can also build and decorate your home as you earn money to expand it. You can also interact with the anthropomorphic animal villagers, giving them gifts, talking to them, and helping to find lost items. It is a 3D game that employs simple, cute character design and world design. It also helps to form a sense of community, as its multiplayer function allows players to visit each other's islands, share unique crops, and interact. It helps build its relaxed, non-competitive game play and emphasis on routine. Though, this game relies on simplicity and peacefulness, it still incorporates subtle elements of haptic feedback to enrich the immersion of the player. Though it might not be considered an "action" game in the traditional sense, it uses haptics to enhance the tactile and emotional connection with the virtual world.

The popularity of \textit{Animal Crossing}\cite{animalcrossing2020} also shows how large the audience is for cozy games. There were 22.4 million copies sold worldwide in the first four months of its release during the pandemic and sales have continued to increase in present day by more than double\cite{Statista2024}.

In \textit{Animal Crossing: New Horizons}\cite{animalcrossing2020}, the Nintendo Switch utilizes HD Rumble technology, which provides vibrations to simulate various actions, such as the sensation of fishing or shaking trees. These small, tactile cues, though subtle, serve to enhance the sense of immersion and gentleness within the virtual environment to be felt outside of the screen. Players report that the feeling of interaction with the environment, such as the vibrations felt when digging up fossils or catching bug, helps to foster a sense of accomplishment and satisfaction.

You can also visit other people's islands and have them visit you, which fosters a sense of community as you can share your accomplishments and gather new materials from your friends. There are also built-in villagers that randomly join your island which help to foster a sense of real-world randomness and surprise. However, they are limited in diversity of interactions and surprises that would break the player out of the game's intended peaceful environment.

\textit{No Man's Sky}\cite{nomanssky2016} is an action, survival game that utilizes haptics, but is included in this paper because it is also seen as cozy by its players due to its feature to explore and incur sense of accomplishments by developing areas. This game supports the PS5 controller's adaptive triggers and can be played with a PlayStation VR2. It is less cartoonish and more complex in its character designs than \textit{Animal Crossing}\cite{animalcrossing2020} but still provides a cozy atmosphere for its players because of the freedom for players to explore and build throughout its multiple galaxies. When you fly the aircraft, you can feel the triggers resist your touch. And if you were to experience with a VR headset, you would feel very immersed as you can see, feel and hear the effects of your actions. It also has many subtle rumbles when you complete an action like boosting your aircraft to visit different planets quicker. Throughout the game, with the PS5 controller, there are different variances and different parts of the pads affected by the rumbles to illustrate feedback. When you jump and then land on a platform, you can feel a slight rumble. When you build, it provides a more satisfying and responsive interaction, as you can feel the controller emit different sizes and placements of its vibrations. With this variance in response, the players are able to feel immersed in the virtual world. This shows that diversity in actions and haptic feedback work even in a game with a cozy genre. However, it may be important to note that the level of aggression in feedback response may need to be toned down for cozy games in comparison to action games.

\section{Methods and Evaluation}
For my mini-project, I wanted to work on the technical aspects of a cozy game, which revolves around building one. I wanted to explore the technical and visual aspects of my project, so that I may have a more realistic expectation of my Senior Comps project and its deadlines. I also wanted to experience any issues that may come up during the process and try to handle them at this simpler level rather than come across them when the process becomes more complicated and the deadline of presentations are near. I wanted to use this experience to gain a more foundational understanding of game design and creation, and extend this process into my final Senior Comps project, "Fishi Feels", which implements an increase use of haptics in cozy games. From the research I conducted through this paper, action games often use more haptic feedback mechanisms (and in greater aggressions). However, this may be due to the inherent nature of the game, which is different from cozy games. Nevertheless, from examining the experience of \textit{No Man's Sky}\cite{nomanssky2016}, the increase use of haptics can increase user experience and immersion even in more calm, slow-paced games. So, because I was able to research and understand the balance to best implement more haptics in cozy games, and the reasons that would affect the seamlessness of the integration, I wanted to attempt this project. In the future, I would like to explore and implement game mechanisms related to fishing, mining, and building and see how the different actions affect the concept of reward for players. This Senior Comps project would aim for the user to build structures based on gathered materials and examine how haptic feedback affects the measured sense of achievement, relaxation, and immersion. So, based on the cozy game genre, haptic feedback, and technical background required for game-making, I had 2 goals in mind for this mini-project: create a peaceful open-world with a controllable character, and find what type of character the users would like to play to help with their sense of immersion.

\subsection{Methods}
The applications used in this mini-project were Blender and Unity. Blender is a popular free open-source application that allows users to create 2D and 3D projects with features like modeling, rigging, animation, and more. It was what I chose to use to create my 3D model for my game and its animations. I chose it because of its useful features like rigging and animation and existing documentation and tutorials. Unity is a popular cross-platform game engine that allows developers to create 2D and 3D games. I chose to use this game engine over others to create my game because it also had features that matched the simple needs of my projects and sufficient amount of documentation and tutorials. It also had an asset store where I could download free assets and had a user-friendly interface for a beginner like myself.

To achieve my goals, I began a character model in blender, focusing on a design that suited a cozy game environment. I followed different tutorials and combined knowledge across different videos to create a model that I felt encapsulated a cozy game. I chose to model the character as short and with more round features rather than sharp. Then, I created an armature for my model to begin creating animation clips to use in my project. Using an armature in Blender also allowed the animations to be transferred to Unity more seamlessly. I wanted to implement animation into this game because an important aspect of haptics and engagement in games is to have the user feel their actions. So, I had created the character to have walking and jumping animations based on the user's controls. I wanted to further expand the types of animations to increase connection between the user and the character, but because of the time limitations, I chose to go onto the next step of my process. I, then created a world in Unity with a terrain that used multiple existing assets to implement ground texture, grass, trees, mushrooms, rocks, and a sky. I went through different tutorials to manage the movement of the character and creation of a diverse environment. I also altered the camera object (the player's view of the game) to follow the character. It took some time to adjust the right distance and angle. I thought about possibly allowing the user to control the camera angle and increasing the interactions to the game, but I wanted to spend more time on creating more interactions with the game's environment. I spent a couple days trying to add water, water texture, and sand, but there was a conflict in assets and the game engine, so I sadly could not include it for this project. I, then created a short survey for 5 individuals to see what type of character they would be attracted to for a cozy game and why. I wanted to examine the features of a character that people were drawn to and in which contexts were they prioritized. For this, the participants were given a blank outline of a person, in which they were allowed 5 minutes to draw a character they would like to control after viewing the Unity world that I developed. I wanted to see what type of realism/style they would give their character based on their perceptions of the world. I, then allowed the users to play the game and give feedback on what they felt and experienced.


\subsection{Evaluation}
For goal 1, I believe that I achieved it but not in the way that I hoped. Because my computer had trouble with its drive to export the blender files correctly, I had to instead use a non-characterized 3d model in Unity as my character. I was still able to create an open world where the character was able to move around, jump, and interact with its environment but not with a character that I created. Nevertheless, the goal was accomplished, and I hope to conduct another experiment where I can compare the differences between a controller and a mouse. It was a goal I had in mind, but my computer had another drive issue with the Controller mapping extension that was found for MAC. So, for this experiment, only the keyboard was used.

For goal 2, the survey showed that users like to choose characters that resemble themselves for cozy games, or at the least have some meaning to them whether it was personal preference/affection or if it resembled someone they liked. 3/5 chose to draw their characters like themselves and 5/5 created a character they liked. I would like to further explore the difference between the emotional connection that one has with their character and see how that affects the amount of time they engage in a game. It was also interesting to note the amount of detail that was drawn in these experiments. All of them drew their character in a cartoonish way with simple, but unique features to distinguish the character as an individual. This may be influenced due to the limited artistic ability of the individual and time constraint, but it provided me some insight on how their creative ability/expectations may influence how connected they feel to games. It provided further insight on how customization and control is important in engagement/immersion. Some surprising observations and feedback from when people were invited to play the game were that the participants were focused on the environment and what it contained more than they were on their character and its interactions. Granted, the character did not have any action that they could commit other than moving/navigating a mountainous terrain, but it was still surprising to see. The only times that they would mention the character or notice it was when it had trouble going up a certain part of the terrain or if it had an amusing fall. The more they noticed the variety in the environment and movement in the character, the more engaged they were. It was interesting to note the difference in attention to environment compared to the character and I hope to further explore it in my Senior Comps project.

\section{Ethical Considerations}
There are many different types of ethical considerations to be made in regards to the entertainment industry with certain motifs in sandbox games that include themes of colonialism. I believe that it is important for developers to acknowledge the game mechanisms that occur when allowing users to cultivate space and the way it impacts the environment. Since games allow for a more engaged interaction by providing a physical sense and continuous new entertainment, it can give rise to issues with players with certain tendencies, and affect their real-world lives. There is also a limitation in representation within games that are made to simulate the real world. This could include race, age, gender, sexuality, and others. Along with a lack of representation, there is also a lack of accessibility. Whether it is visual, auditory, or cognition-related, the industry does not entirely consider issues of barriers, and consequently limits certain groups from participating in games. With haptic feedback continuing to be developed, it is important for the industry to acknowledge now the effects of their game design and have consideration for the people using it.

\subsection{Colonialism and Over Consumption}
With the growing amount of people that frequent cozy games, a medium often used for relaxation and a sense of accomplishment, it is important to note the goals of the games played and how game mechanisms are developed and used by developers. When these games have a narrative that the player is interacting with "empty lands", and is encouraged to cultivate, it can seem to resemble familiar narratives of colonialism, and its erasure of those native, and view it as good \cite{USA_Smith_2022}. So though, sandbox games may seem very innocent due to their cozy and non-competitive nature, they can reinforce ideas of colonialism and over consumption with a lack of proper consequences or acknowledgment. Some sandbox games allow for the users to speed up certain processes of natural phenomena like trees bearing fruits, so that they can use the fruit to receive money or other game benefits. This allows the user to exploit as much as they want as long as they put in the time to. This is dangerous as games may be encouraging for users to not only over consume in the games, but also in real-life with the time and energy they devote to their virtual worlds. With sandbox games, it relies on a reward system where different behaviors/mechanisms will provide the player with some sort of reward. Due to this randomness, variety, and idea of reward, it can provide dopamine and be addictive to players\cite{Staewen_Trevino_Chang_Yun_2014}. This goal-oriented playing and repetition for users can shape behavior and also mentality. Especially for cozy games, where routine is a key concept of the game mechanisms. Studies have found that people prefer non-randomized rewards for games, which might show the addiction towards cozy games.\cite{Staewen_Trevino_Chang} This harms most people, but especially those who lack behavioral control (seen in those with ADHD) and lack social relationships\cite{Eijnden_Peeters_2024}. The games may be taking advantage of these individuals to retain player engagement and sources of income for the companies. Utilizing different reward systems by asking for a certain amount of activity for reward or asking for certain tasks to be done in a certain amount of time, encourages the players to continue to play and gather more resources for their virtual developments and sense of accomplishment. Those with less social interactions outside of virtual worlds may indulge further because they want to have lasting social interactions through playing with Non-Player Characters (NPCs) or online friends that they may not have in their daily lives.\cite{Eijnden_Peeters_2024}

\subsection{Character Limitations}
Though we are seeing a stronger push for representation in current media, there is still a lot to be seen in games. In a 2009 paper published by \textit{New Media and Society}, they found that there is an over-representation of those white, 18+, and male, while there is less representation for those who are female, Hispanics, Native Americans, children and elderly. At the time of the paper, the results were similar to the representation in TV. Though this paper is 16 years old, it still has truth today. People are still fighting for racial, gender and sexual orientation representation. Additionally, rarely, do you see a game where you can play as an elderly person or as someone disabled. These are great populations without representations that do not see themselves when they play games. Without seeing themselves in a medium that is known to encourage exploration and a sense of relaxation and accomplishment, it causes a lack of acknowledgment for people's capabilities and existence. We can also still see gender bias through games with the amount of dialogue given to female characters compared to male characters, which is almost half (though this could be due to the lack of female characters in games overall ).\cite{Rennick_Clinton_Ioannidou_Oh_Clooney_T._Healy_Roberts}

Character customization is a unique aspect of games where you can shape the character you play, creating a stronger emotional bond to the game. Though the complexity of this feature ranges between games, it usually comprises, at the least of, skin color, hair, eye shape, eye color, nose, smile and clothes. Recently, we have been seeing more diversity in skin color to include more darker shades in simpler design characters like \textit{Animal Crossing}, but we have more to see with age and disability.

\subsection{Accessibility}
As industries in a profit-driven environment continue to innovate for stockholders and the growing market, they may move forward and leave certain demographics behind. Games provide power and comfort to individuals, but if certain mechanisms work against their cognition or embodiment, they cannot enjoy the future innovations that rest on it like the others. With cozy games that focus on simulation and Non-Player Characters, there is a lot of dialogue and texts. This may be challenging for those with visual impairments, dyslexia or ADHD.\cite{Gaming_Read_2022} 

The gaming industry has been making strides to increase accessibility and allow a broader audience to explore gaming, where major titles offer different customizations to accommodate various needs. Unity plugins allow developers to integrate text-to-speech functionality into games, and also allow players to select certain areas for their text dialogues. However, many games ultimately do not have many accessible options. Providing an option for text-to-speech, larger fonts, and/or fewer texts at a time can help improve these individuals' enjoyment of the game. There are other instances where changes can help an individual's experience including changing certain color schemes to help with the color blindness or including subtitles for dialogue and sounds for those hard of hearing.\cite{Hassan_2024} 

Challenges remain within this industry to increase accessibility and accommodations. Some may implement features, but not at the quality that people need. It is important for developers to pay attention to prioritizing inclusivity, and increasing diversity within player needs, especially during the design and testing phases. As the technical field continues to trek forward in innovation, it is important to recognize the accessibility of their products and how to be inclusive in design.

\section{Timeline}
This is my proposed timeline for the Senior Comprehensive Project:

Summer (tentative):
\begin{itemize}
\item Narrow down a realistic game design and skills needed for project
\item Research tutorials on game development
\item Map the controls of a controller to a simple game
\end{itemize}

August:
\begin{itemize}
\item Research/survey how users feel connected to their controlled characters
\item Research existing haptic mechanisms that help user experience and performance
\item Create movable character in 3D world
\end{itemize}

September:
\begin{itemize}
\item Create game mechanisms that implement a rumble system
\item Debug/implement game mechanics for users
\end{itemize}

October:
\begin{itemize}
\item First formal user testing
\item Consolidate and analyze data
\item Continue to develop and debug game
\end{itemize}

November:
\begin{itemize}
\item Begin creating poster
\item Finalize game and test
\end{itemize}

December:
\begin{itemize}
\item Finish poster and presentation materials
\item Finalize report
\end{itemize}


\appendix

\printbibliography

\end{document}
